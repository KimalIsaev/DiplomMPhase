Сделаем подстановки \eqref{first_changes} в 
\eqref{HFirstEquationParticular} и \eqref{HSecondEquationParticular}:
\begin{equation}\begin{aligned}
\label{FFirstEquationParticular}
\varepsilon \pdv{\Fn(w,\tau,\varepsilon)}{\tau}=
    &-\Q{\lambda+\allExecutionFullIntensity{n}{k}}\seqStandart{F}{n}(w,\tau,\varepsilon)+\\
    &+j \En \pdv{\seqStandart{F}{n}(w,\tau,\varepsilon)}{w}(i,\tau) + \\
    &+\overline{\En} \lambda e^{j\varepsilon w} \seqStandart{F}{n}(w,\tau,\varepsilon)+\\
    &+r_1 \Q{\allExecutionRelativeIntensity{n}{k}} \seqStandart{F}{n}(w,\tau,\varepsilon) + \\
    &+\lambda\sum_{k=0}^{M-1}q_k \seqMinus{F}{n}{k}(w,\tau,\varepsilon) + \\
    &-j  e^{-j\varepsilon w} \sum_{k=0}^{M-1}q_k \pdv{\seqMinus{F}{n}{k}(w,\tau,\varepsilon)}{w} + \\
    &+\En r_0 \sum_{f=0}^{M-1} \mu_f (n_f + 1) \seqPlus{F}{n}{f}(w,\tau,\varepsilon) + \\
    &+\En r_2 e^{j\varepsilon w} \sum_{f=0}^{M-1} \mu_f (n_f + 1) \seqPlus{F}{n}{f}(w,\tau,\varepsilon) + \\
    &+r_1 \feedbackSum{f}{k}
        q_k \mu_f (n_f+1) \seqMinusPlus{F}{n}{k}{f}(w,\tau,\varepsilon),
\end{aligned}\end{equation}

\begin{equation}\begin{aligned}\label{FSecondEquationParticular}
\varepsilon \nJ \pdv{\Fn(w,\tau,\varepsilon)}{\tau}=(e^{j \varepsilon w} - 1)
\Bigg\{
    &\lambda \nI \seqStandart{F}{n}(w,\tau,\varepsilon)+\\
    &+j e^{-j \varepsilon w} \nG\pdv{\seqStandart{F}{n}(w,\tau,\varepsilon)}{w} + \\
    &+r_2\sum_{f=0}^{M-1} \mu_f \nJ n_f \seqStandart{F}{n}(w,\tau,\varepsilon)
\Bigg\}
.
\end{aligned}\end{equation}

При условии \(\varepsilon \rightarrow 0\), можно
доказать следующие утверждение.

Теорема.2.1 Компоненты вектора \(\Rn\) распределения вероятнойстей
числа приборов, занятых на соответствующих фазах имеет вид:
\[\Rn = \frac{\Ln}{\sum_{\bar{m}}^{\Jt}\Lm},\]
где
\[\Ln = \Ldefinition{n}{d}.\]

Доказательство.
