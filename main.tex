\documentclass[a4paper, 12pt]{article}
%\usepackage[OT2, T1]{fontenc}
\usepackage[left=3cm,right=1.5cm,top=2cm,bottom=2cm]{geometry}
\usepackage{amsmath}
\usepackage{amssymb}
\usepackage{minted}
\usepackage[utf8]{inputenc}
%\usepackage[T1, T2A]{fontenc}
\usepackage[unicode]{hyperref}
\usepackage[russian, english]{babel}
\usepackage{comment}
\usepackage{subfiles}
\usepackage{pxfonts}
\usepackage{environ}
\usepackage{changepage}
\usepackage{physics}
\usepackage[fleqn,tbtags]{mathtools}

\usepackage{titletoc}
\usepackage{titlesec}
\usepackage{titleps}
%\usepackage[no-math]{fontspec}
\usepackage{fontspec}%
\newfontfeature{Microtype}{protrusion=default;expansion=default;}%
%\setmainfont[Ligatures=TeX]{Iwona Cond}
\setmainfont{times}[
  Extension      = .ttf ,
  BoldFont       = *bd ,
  ItalicFont     = *i ,
  BoldItalicFont = *bi
  %Microtype,
  %Ligatures = TeX,
  %BoldFont = timesbd.ttf,
  %ItalicFont = timesi.ttf,
  %BoldItalicFont = timesbi.ttf,
]
\usepackage[none]{hyphenat}
\sloppy
\hyphenation{ин\-тен\-сив\-ность ин\-тен\-сив\-нос\-тя\-ми}

\usepackage{lipsum}
\usepackage{float}
\setlength\parindent{12.5mm}

\setlength{\parskip}{1em}

\addto\captionsrussian{% Replace "english" with the language you use
    \renewcommand{\contentsname}%
    {\centering{\normalsize{ОГЛАВЛЕНИЕ}}}%
}
\addto\captionsrussian{% Replace "english" with the language you use
    \renewcommand{\figurename}{Рисунок}
}

\usepackage{setspace}
\onehalfspacing

\usepackage[nooneline]{caption}
\captionsetup[figure]{justification=centering,labelsep=endash}
\captionsetup[table]{justification=centering,labelsep=endash}

\DeclareSymbolFont{matha}{OML}{txmi}{m}{it}
\DeclareMathSymbol{\varv}{\mathord}{matha}{118}

\newcommand{\Q}[1]{\left( {#1} \right)}
\newcommand{\T}[1]{\left[ {#1} \right]}
\newcommand{\N}[1]{\left\{ {#1} \right\}}

\newcommand{\kronm}{\delta_{i,-1}}
\newcommand{\kronzero}{\delta_{i, 0}}
\newcommand{\kronplus}{\delta_{i,1}}
\newcommand{\Jt}{\mathbb{J}}
\newcommand{\It}{\mathbb{I}}
\newcommand{\Gt}{\mathbb{G}}
\newcommand{\Zt}{\mathbb{Z}}
\newcommand{\Mt}{[1,M]\subset\Zt}
\newcommand{\Nt}{[0,N]\subset\Zt}

\newcommand{\Pn}{{P}_{\bold{n}}}
\newcommand{\Hn}{{H}_{\bold{n}}}
\newcommand{\Hln}{{H}^{(1)}_{\bold{n}}}
\newcommand{\Fln}{{F}^{(1)}_{\bold{n}}}
\newcommand{\Fn}{{F}_{\bold{n}}}
\newcommand{\gn}{{g}_{\bold{n}}}
\newcommand{\pn}{{\varphi}_{\bold{n}}}
\newcommand{\En}{E_{\bold{n}}}
\newcommand{\Ln}{L_{\bold{n}}}
\newcommand{\Rn}{R_{\bold{n}}}
\newcommand{\Lk}{L_{\bold{k}}}
\newcommand{\Rk}{R_{\bold{k}}}
\newcommand{\Lm}{L_{\bold{m}}}
\newcommand{\Rm}{R_{\bold{m}}}

\newcommand{\nJ}{\sum_{\bold{n}}^{\Jt}}
\newcommand{\nG}{\sum_{\bold{n}}^{\Gt}}
\newcommand{\nI}{\sum_{\bold{n}}^{\It}}

\newcommand{\dt}{
    \Delta t
}
\newcommand{\lx}{
    \lambda + x
}
\newcommand{\lxt}{
    \lambda + x(\tau)
}
\newcommand{\ls}{
    \lambda + i\sigma
}
\newcommand{\les}{
    \lambda + \En i \sigma
}
\newcommand{\LunityBase}{
    \frac{\lxt}{1-r_1}
}
\newcommand{\Lunity}[1]{
    \Q{\LunityBase}^{#1}
}
\newcommand{\LunityStandart}[2]{
    \Lunity{
        \sum_{#2=1}^{M}{#1}_{#2}
    }
}
\newcommand{\executionAbsoluteIntensity}[1]{
    \frac{\mu_{#1}}{q_{#1}}
}
\newcommand{\executionAbsoluteReverseIntensity}[1]{
    \frac{q_{#1}}{\mu_{#1}}
}
\newcommand{\LuniqueBase}[2]{
        \frac{1}{{#1}_{#2}!}
        \Q{\executionAbsoluteReverseIntensity{#2}}^{{#1}_{#2}}
}
\newcommand{\LuniqueStandart}[2]{
    \prod_{{#2}=1}^{M}
    \LuniqueBase{#1}{#2}
}

\newcommand{\Ldefinition}[2]{
    \LunityStandart{#1}{#2}
    \LuniqueStandart{#1}{#2}
}
\newcommand{\LminusDefinition}[3]{
    \Lunity{\sum_{{#2}=1}^{M}{#1}_{#2} - 1}
    \T{\prod_{{#2}=1}^{M}\LuniqueBase{#1}{#2}}
    \Q{{#1}_{#3}\executionAbsoluteIntensity{{#3}}}
}
\newcommand{\seqStandartStar}[2]{
    {#1}_{({#2}_1, {#2}_2, ... {#2}_{M})^*}
}
\newcommand{\seqMinusStar}[3]{
    {#1}_{({#2}_1, {#2}_2, ... , {#2}_{#3} - 1,  ... {#2}_{M})^*}
}
\newcommand{\seqPlusStar}[3]{
    {#1}_{({#2}_1, {#2}_2, ... , {#2}_{#3} + 1,  ... {#2}_{M})^*}
}
\newcommand{\seqMinusPlusStar}[4]{
    {#1}_{({#2}_1, {#2}_2, ... , {#2}_{#3} - 1, ... , 
        {#2}_{#4} + 1, ... {#2}_{M})^*}
}
\newcommand{\seqStandart}[2]{
    {#1}_{{#2}_1, {#2}_2, ... {#2}_{M}}
}
\newcommand{\seqMinus}[3]{
    {#1}_{{#2}_1, {#2}_2, ... , {#2}_{#3} - 1,  ... {#2}_{M}}
}
\newcommand{\seqPlus}[3]{
    {#1}_{{#2}_1, {#2}_2, ... , {#2}_{#3} + 1,  ... {#2}_{M}}
}
\newcommand{\seqMinusPlus}[4]{
    {#1}_{{#2}_1, {#2}_2, ... , {#2}_{#3} - 1, ... , 
        {#2}_{#4} + 1, ... {#2}_{M}}
}
\newcommand{\allExecutionFullIntensity}[2]{
    \sum_{{#2}=1}^{M}\mu_{#2}{#1}_{#2}
}
\newcommand{\allExecutionRelativeIntensity}[2]{
    \sum_{{#2}=1}^{M} q_{#2}\mu_{#2}{#1}_{#2} 
}
\newcommand{\feedbackSum}[2]{
    \sum_{{#1}=1}^{M} \sum_{{#2}=1, {#2} \neq {#1}}^{M}
}
\DeclareMathOperator*{\argmax}{arg\,max}
\NewEnviron{urv}[1]{\begin{adjustwidth}{-#1pt}{0pt}\begin{align*} \BODY \end{align*}\end{adjustwidth}}
\begin{document}
    \subfile{titulList}
    \newpage
    \setcounter{page}{2}
    \subfile{tasks}
    \begin{center}
        \bf{АННОТАЦИЯ}
    \end{center}
    \subfile{anotaciya}
    \newpage

\setcounter{secnumdepth}{0}
\dottedcontents{section}[3.8em]{}{2.3em}{1pc}
\dottedcontents{subsection}[5em]{}{2.3em}{1pc}
    \tableofcontents
    \newpage

\titleformat*{\section}{\bf \center \uppercase}
 \(\text{ }\)\\

\sloppy
\section{Введение}

    \subfile{introduction}
    \newpage

\titleformat*{\section}{\bf  \filright \uppercase}
\titlespacing{\section}{12.5mm}{*1}{*1}
%\titlespacing{\section}{5pt}{*4}{*1.5}
\titleformat*{\subsection}{\bf \filright }
\titlespacing{\subsection}{12.5mm}{*1}{*1}
\section{1 Исcледование марковской системы массового обслуживания с постоянными интенсивностями перехода между состояниями, за исключением интенсивности ухода заявок с орбиты, с общей орбитой, на которую заявки приходят и с которой заявки уходят ординарно, в условиях стремящейся к бесконечности задержки на орбите}
\subsection{1.1 Получение системы дифференциальнных уравнений в частных производных}
\setlength\parindent{12.5mm}

    \subfile{norm}\\\\
    \subfile{conditions}
    \newpage
    \subfile{kolmogorov_before_differential}
    \newpage
    \subfile{kolmogorov_differential}
    \subfile{kolmogorov_after_differential}
    \subfile{summa_na_edinich_nol}
    \subfile{pre_partial_case}
    \newpage
\subsection{1.2 Получение аппроксимации решения системы дифференциальных уравнений}
    \setlength\parindent{12.5mm}
    \subfile{stationary_probability_theorem}
    \subfile{a}
    \subfile{theorem14_description}
    \subfile{theorem14_proof}
    \newpage
\section{2 Исследование M|$H_M$|N с общей орбитой.}
    \subfile{sets_in_particular}
    \subfile{P_equation_in_particular}
    \subfile{H_equation_in_particular}
    \subfile{F_equation_in_particular_description}
    \subfile{F_equation_in_particular_proof}
    \subfile{suppose_L_in_particular}
    \subfile{L_equation_in_particular}
    \subfile{x_equation_in_particular}
    \subfile{proof_for_proof}
\section{3 Нахождение программными средствами аппроксимации распределения заявок на орбите}

    \subfile{usl}
    \newpage
	\subfile{library_code}
    \newpage
	\subfile{example_code}
    %\subfile{introduce_set_v.tex}
    \newpage
\end{document}
