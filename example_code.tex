Подключаем необходимые библиотеки, ставим печать
с помощью ASCI символов для более удобного чтения
полученных файлов в будущемЖ
\begin{minted}{Python}
import itertools
from lib import algorithm
from sympy import *
from pathlib import Path
from operator import mul
from math import factorial
from functools import reduce
import sys
init_printing(use_unicode=False)
\end{minted}

Определяем функцию, которая выдаёт функцию, что
возращает логическую единицу, только тогда, когда
сумма элементов меньше чем значение переданное в изначальную
функцию: 
\begin{minted}{Python}
def sum_checker(M):
    def f(v):
        return (sum(v) < M)
    return f
\end{minted}

Эта функция возращает множество \(\Jt\), принимая
количиство элементов в элементе \(\Jt\),
максимум в элементе вектора принадлежащего \(\Jt\) плюс один,
и функцию, что определяет что входит в множество \(\Jt\):
\begin{minted}{Python}
def check_all_states(N, M, f):
    return [v for v in itertools.product(range(M), repeat=N) if f(v)]
\end{minted}

Функция, что выдаёт все состояния системы массового обслуживания
для N приборов и M-1 фаз:
\begin{minted}{Python}
def states_for_phase_execution(N, M):
    return check_all_states(N, M, sum_checker(M))
\end{minted}

Функция для создания функции циферной нумеровки:
\begin{minted}{Python}
def create_my_numerate(M):
    def my_norm(x):
        return sum([k*M**i for i, k in enumerate(x)])
    return my_norm
\end{minted}

Функция, что выдаёт все состояния системы массового обслуживания
для N приборов и M фаз, отсортированные в нужном порядке,
используя циферную функцию нумеровки.
\begin{minted}{Python}
def get_list_of_states(N, M):
    return sorted(states_for_phase_execution(N, M+1),
        key=create_my_numerate(M+1))
\end{minted}

Функция создающая функцию, что отдаёт на
каком элементе вектора первый вектор больше второго
на единицу и/или меньше его, или же равенство онных.
\begin{minted}{Python}
def create_ordinar_difference_of_vector(N):
    def f(z, y):
        i_minus = N
        i_plus = N
        for i, (m, k) in enumerate(zip(z, y)):
            diff = m - k
            if abs(diff) > 1:
                return (N, N), False
            if diff == 1:
                if i_plus == N:
                    i_plus = i
                else:
                    return (N, N), False
            if diff == -1:
                if i_minus == N:
                    i_minus = i
                else:
                    return (N, N), False
        return (i_minus, i_plus), True
    return f
\end{minted}

Функция создающая функцию, что отдаёт 
локальные аргументы для вычисления матриц.
\begin{minted}{Python}
def create_phase_execution_local_arg_maker(N):
    diff_f = create_ordinar_difference_of_vector(N)

    def create_local_arg(q, y, z, const_arg):
        (i_minus, i_plus), check = diff_f(z, y)
        no_minus = (i_minus == N)
        no_plus = (i_plus == N)
        is_minus = not no_minus
        is_plus = not no_plus
        no_change = (q == 0)
        plus_change = (q == 1)
        minus_change = (q == -1)
        arg = {'i_minus': i_minus, 'i_plus': i_plus,
               'no_minus': no_minus, 'no_plus': no_plus,
               'is_minus': is_minus, 'is_plus': is_plus,
               'check': check, 'no_change': no_change,
               'plus_change': plus_change,
               'minus_change': minus_change}
        return arg

    return create_local_arg
\end{minted}

\begin{minted}{Python}
def create_phase_execution_LFL(M):
    def poiss_checker(q, y, z, const_arg, arg):
        right_state_change = arg['no_minus'] and arg['is_plus']
        return arg['check'] and arg['no_change'] and right_state_change

    def poiss(q, y, z, const_arg, arg):
        qk = const_arg['q'][arg['i_plus']]
        lam = const_arg['lam']
        return lam*qk

    def out_orbit_checker(q, y, z, const_arg, arg):
        right_state_change = arg['no_minus'] and arg['is_plus']
        return arg['check'] and arg['minus_change'] and right_state_change

    def out_orbit(q, y, z, const_arg, arg):
        return 0

    def out_orbit_(q, y, z, const_arg, arg):
        qk = const_arg['q'][arg['i_plus']]
        sigma = const_arg['sigma']
        return sigma*qk

    def in_orbit_checker(q, y, z, const_arg, arg):
        right_state_change = arg['is_minus'] and arg['no_plus']
        return arg['check'] and arg['plus_change'] and right_state_change

    def in_orbit(q, y, z, const_arg, arg):
        k = arg['i_minus']
        uk = const_arg['u'][k]
        r2 = const_arg['r2']
        return y[k]*uk*r2

    def in_orbit_full_checker(q, y, z, const_arg, arg):
        right_state = arg['no_minus'] and arg['no_plus'] and (sum(y) == M)
        return arg['check'] and arg['plus_change'] and right_state
    def in_orbit_full(q, y, z, const_arg, arg):
        return const_arg['lam']

    def exit_checker(q, y, z, const_arg, arg):
        right_state_change = arg['is_minus'] and arg['no_plus']
        return arg['check'] and arg['no_change'] and right_state_change

    def exit_zero(q, y, z, arg_dict, temp_arg_dict):
        k = temp_arg_dict['i_minus']
        uk = arg_dict['u'][k]
        r0 = arg_dict['r0']
        return y[k]*uk*r0

    def again_checker(q, y, z, const_arg, arg):
        right_state_change = arg['is_minus'] and arg['is_plus']
        return arg['check'] and arg['no_change'] and right_state_change

    def again(q, y, z, const_arg, arg):
        k = arg['i_minus']
        uk = const_arg['u'][k]
        r1 = const_arg['r1']
        return y[k]*uk*r1*const_arg['q'][arg['i_plus']]

    return [[poiss_checker, poiss], [out_orbit_checker, out_orbit, out_orbit_],
            [exit_checker, exit_zero], [in_orbit_full_checker, in_orbit_full],
            [in_orbit_checker, in_orbit], [again_checker, again]]
\end{minted}



