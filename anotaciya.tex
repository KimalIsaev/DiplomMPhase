\hspace*{\parindent}%
Настоящая работа посвящена исследованию системы массового обслуживания с простейшим потоком, орбитой, а также c произвольным количеством приборов, гиперэкспоненциальным распределением времени обслуживания c произвольным количеством фаз и с обратной связью.

\textbf{Ключевые слова:} теория массового обслуживания, система массового обслуживания, RQ-система, характеристическая функция, метод асимптотического анализа, метод асимптотически диффузионного анализа, простейший поток, гиперэкспоненциальное распределение, произвольное количество приборов, произвольное количество фаз, орбита, имитационное моделирование.

\textbf{Объект исследования:} RQ-система $M|H_M|N$.

\textbf{Цель:} построить ряд распределения вероятностей, или его аппроксимацию, количества заявок на орбите для RQ-системы $M|H_M|N$ в стационарном режиме.

\textbf{Структура работы:} настоящая работа включает в себя 3 раздела, 55 страницы, 9 рисунков, 21 источник литературы.

Рассматривается система массового обслуживание с простейшим входящим потоком, интенсивность которого равняется \(\lambda\), с общей орбитой, на которой интенсивность каждой заявки равна \(\sigma\), и N приборов, каждый из которых имеет гиперэкспоненциальное М-фазное распределения с вероятностями перехода на k-ую фазу \(q_k\), где \(\sum_{k=1}^M q_k = 1\), а время пребывания заявки на k-ой фазе распределено по экспоненциальному закону с параметром \(\mu_k\), при окнчании обслуживания с вероятность \(r_0\) заявка выходит из системы, с вероятностью \(r_1\) поступает на мгновенное повторное обслуживание, где снова определяется фаза для обслуживания, с вероятностью \(r_2\) поступает на общую орбиту для отложенного повторного обслуживания, причём \(r_0 + r_1 + r_2 = 1\).

В результате исследования получен алгоритм вычисления аппроксимации распределения количества заявок на орбите в стационарном режиме для любой марковской RQ-системы массового обслуживания с общей орбитой при заданых постоянных интенсивностях изменения состояния системы, кроме интенсивности убывания заявок с орбиты, что должна быть линейно зависима от количества заявок на орбите, с нулевым свободным коэффицентом, при ординарном поступлении заявок на орбиту и ординарном убывании заявок с орбиты. После чего доказывается применимость данного алгоритма для системы массового обслуживания описанной выше и сравниваются полученные результаты с результатами имитационного моделирования.


