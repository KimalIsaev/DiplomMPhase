\[P_{\bold{y}^*, \bold{z}^*}( i, l, t, h) = P\{\bold{z}^* = \bold{y}^*(t + h), l = i(t + h)\text{ }|\text{ } \bold{y}^* = \bold{y}^*(t), i = i(t) \},\]
\[P_{\bold{y}^*}(i, t) = P_{0, \bold{y}^*}( 0, i, 0, t).\]
Где \(\bold{y}\)(t) \(\in \mathbb{V}\) - вектор состояния системы массосвого обслуживания, а i(t) - количество заявок на орбите. 
\( t,h \in [0; + \infty) \subset \mathbb{R}\).\\
Теорема 1.1:
Если \(\exists \mathbb{J} \subset \mathbb{V}\), такое что:
\begin{equation}\label{condition_of_main_theorem}\begin{aligned}
		1.& \forall \bold{y} \in \mathbb{J}^\complement = \mathbb{V} \setminus{} \mathbb{J},
		\forall \bold{z} \in \mathbb{V}, \forall i,l\in [0,+\infty)\subset \mathbb{Z}, 
		\forall t,h \in [0;+\infty) \subset \mathbb{R}: \\
		&P_{ \bold{y}^*, \bold{z}^*}( i, l, t, h)  = P_{\bold{z}^*, \bold{y}^*}( i, l, t, h) =  0.\\
		2.& \text{При очень малых h, } \forall \bold{y},\bold{z} \in \mathbb{J}: \\
			&P_{ \bold{y}^*, \bold{z}^*}(i, i, t, h) = C_{\bold{y}^*, \bold{z}^*}; \\
			&P_{ \bold{y}^*, \bold{z}^*}(i, i+1, t, h) = B_{\bold{y}^*, \bold{z}^*}; \\
			&P_{ \bold{y}^*, \bold{z}^*}(i, i-1, t, h) = \eta K_{\bold{y}^*, \bold{z}^*}i; \\
			&P_{ \bold{y}^*, \bold{z}^*}(i, i+q, t, h) = 0; \\
            &\forall i \in [0,+\infty)\subset \mathbb{Z}, \forall q \in (-\infty, -1) \cup (1,+\infty) \subset \mathbb{Z}, 
\end{aligned} \end{equation}
Тогда:
\begin{equation}\label{pervoe_uravnenie_partial}
    \pdv{\boldsymbol{H}(u,t)}{t} =
    \boldsymbol{H}(u,t)\bigg[\boldsymbol{C} + e^{ju}\boldsymbol{B} - \boldsymbol{D}\bigg]
    -j\eta\pdv{\boldsymbol{H}(u,t)}{u}\bigg[e^{-ju}\boldsymbol{K} - \boldsymbol{I}\bigg].
\end{equation}
\begin{equation}\label{vtoroe__uravnenie}
    \pdv{\boldsymbol{H}(u,t)}{t}\boldsymbol{e} =
    (e^{ju} - 1) \bigg[
    \boldsymbol{H}(u,t)  \boldsymbol{B}
    + j\eta e^{-ju}\pdv{\boldsymbol{H}(u,t)}{u}\boldsymbol{I} \bigg]\boldsymbol{e}.
\end{equation}
Где,
\begin{gather*}
	\boldsymbol{H}(u,t) = \bigg[H_{\tilde{1}}(u,t), H_{\tilde{2}}(u,t), ... H_{\tilde{|\mathbb{J}|} }(u,t)\bigg] ,\\
	\boldsymbol{C} = ||C_{\tilde{n},\tilde{m}}||,
	\text{ }\text{ }\text{ }\text{ }\text{ }
	\boldsymbol{B} = ||B_{\tilde{n},\tilde{m}}||,
	\text{ }\text{ }\text{ }\text{ }\boldsymbol{D} = \bigg|\bigg|\delta_{n,m}
	\sum_{\nu = 1}^{|\mathbb{J}|}\bigg(C_{\tilde{\nu},\tilde{n}} + B_{\tilde{\nu},\tilde{n}} \bigg)\bigg|\bigg|, \\
	\boldsymbol{K} = ||K_{\tilde{n},\tilde{m}}||,
	\text{ }\text{ }\text{ }\text{ }\boldsymbol{I} = \bigg|\bigg|\delta_{n,m}
	\sum_{\nu = 1}^{|\mathbb{J}|}K_{\tilde{\nu},\tilde{n}}\bigg|\bigg|, \\
	\bold{j}^*_1 < \bold{j}^*_2 < \bold{j}^*_3 <... < \bold{j}^*_{|\mathbb{J}|} < M^n, \text{ }\text{ }\text{ }
	\forall \nu \in [1, |\mathbb{J}|] \subset \mathbb{Z}, \tilde{\nu} = \bold{j}^*_{\nu}.
\end{gather*}

\begin{comment}
\[P_{\bold{y}^*, \bold{z}^*}( i, l, t, h) = P\{\bold{z}^* = \bold{y}^*(t + h), l = i(t + h)\text{ }|\text{ } \bold{y}^* = \bold{y}^*(t), i = i(t) \},\]
\[P_{\bold{y}^*}(i, t) = P_{0, \bold{y}^*}( 0, i, 0, t).\]
Где \(\bold{y}\)(t) \(\in \mathbb{V}\) - вектор состояния системы массосвого обслуживания, а i(t) - количество заявок на орбите. 
\( t,h \in [0; + \infty) \subset \mathbb{R}\).\\
Теорема 1.1:
Если \(\exists \mathbb{J} \subset \mathbb{V}\), такое что:
\begin{equation}\label{condition_of_main_theorem}\begin{aligned}
		1.& \forall \bold{y} \in \mathbb{J}^\complement = \mathbb{V} \setminus{} \mathbb{J},
		\forall \bold{z} \in \mathbb{V}, \forall i,l\in [0,+\infty)\subset \mathbb{Z}, 
		\forall t,h \in [0;+\infty) \subset \mathbb{R}: \\
		&P_{ \bold{y}^*, \bold{z}^*}( i, l, t, h)  = P_{\bold{y}^*, \bold{z}^*}( i, l, t, h) =  0.\\
		2.& \text{При очень малых h } 
            \forall \bold{y},\bold{z} \in \mathbb{J}, 
            \forall i,l\in [0,+\infty)\subset \mathbb{Z}, 
            (\bold{y}^* \neq \bold{z}^*) \& (i \neq l):\\
		    &P_{ \bold{y}^*, \bold{z}^*}(l, i, t, h) =  L^{(l-i)}_{ \bold{y}^*, \bold{z}^*}(i)h + o(h).\\
		    &\text{Где: }L^{(q)}_{ \bold{y}^*, \bold{z}^*}(i)=C^{(q)}_{ \bold{y}^*, \bold{z}^*, 0}+\sum^{N}_{k=1}i^kC^{(q)}_{ \bold{y}^*, \bold{z}^*, k}, 
		    \text{ }C^{(q)}_{ \bold{y}^*, \bold{z}^*, k}\in \mathbb{R}; N \in \mathbb{N}.\\
		    &\text{И, } \forall \bold{y}^*,\bold{z}^* \in [0,M^n -1] \subset \mathbb{Z},C^{(-1)}_{ \bold{y}^*, \bold{z}^*,0} = 0,\\
		    &\text{При }q\in ((-\infty,-1) \cup (W, +\infty)) \subset \mathbb{Z}, 
		    L^{(q)}_{ \bold{y}^*, \bold{z}^*}(i)=0, W \in \mathbb{N}, W < +\infty.
\end{aligned} \end{equation}
Тогда:
\[\pdv{\boldsymbol{H}(u,t)}{t} =\sum_{k = 0}^{N}(-j)^k \pdv[k]{\boldsymbol{H}(u,t)}{u} \bigg[
\sum_{q = -1}^{W} e^{juq}\boldsymbol{C}_{q,k}-\boldsymbol{D}_k\bigg].\]
Где,
\begin{gather*}
	\boldsymbol{H}(u,t) = \bigg[H_{\tilde{1}}(u,t), H_{\tilde{2}}(u,t), ... H_{\tilde{|\mathbb{J}|} }(u,t)\bigg] ,\\
	\boldsymbol{C}_{q,k} = ||C^{(q)}_{\tilde{n},\tilde{m}, k}(1-\delta_{n,m}\delta_{q,0})||,
	\text{ }\text{ }\text{ }\text{ }\text{ }\text{ }\text{ }\text{ }\text{ }\boldsymbol{D}_k = \bigg|\bigg|\delta_{n,m}
	\sum_{\nu = 1}^{|\mathbb{J}|}\sum_{q = -1 ,\atop \tilde{\nu}\neq \tilde{n}\& q\neq 0 }^{W}C^{(q)}_{\tilde{\nu},\tilde{n},k} \bigg|\bigg|, \\
	\bold{j}^*_1 < \bold{j}^*_2 < \bold{j}^*_3 <... < \bold{j}^*_{|\mathbb{J}|} < M^n, \text{ }\text{ }\text{ }
	\forall \nu \in [1, |\mathbb{J}|] \subset \mathbb{Z}, \tilde{\nu} = \bold{j}^*_{\nu}.
\end{gather*}
\end{comment}
