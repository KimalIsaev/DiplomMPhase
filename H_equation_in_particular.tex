Введём характеристическую функцию:
\[\Hn(u,t)=\sum_{i=0}^\infty e^{iuj}\Pn(i,t).\]
\begin{equation*}\begin{aligned}
\pdv{\Hn(u,t)}{t}=
    &-\Q{\lambda+\allExecutionFullIntensity{n}{k}}\seqStandart{H}{n}(u,t)+\\
    &+j \sigma \En \pdv{\seqStandart{H}{n}(u,t)}{u}(i,t) + \\
    &+\overline{\En} \lambda e^{ju} \seqStandart{H}{n}(u,t)+\\
    &+r_1 \Q{\allExecutionRelativeIntensity{n}{k}} \seqStandart{H}{n}(u,t) + \\
    &+\lambda\sum_{k=0}^{M-1}q_k \seqMinus{H}{n}{k}(u,t) + \\
    &-j \sigma  e^{-ju} \sum_{k=0}^{M-1}q_k \pdv{\seqMinus{H}{n}{k}(u,t)}{u} + \\
    &+\En r_0 \sum_{f=0}^{M-1} \mu_f (n_f + 1) \seqPlus{H}{n}{f}(u,t) + \\
    &+\En r_2 e^{ju} \sum_{f=0}^{M-1} \mu_f (n_f + 1) \seqPlus{H}{n}{f}(u,t) + \\
    &+r_1 \feedbackSum{f}{k}
        q_k \mu_f (n_f+1) \seqMinusPlus{H}{n}{k}{f}(u,t).
\end{aligned}\end{equation*}
Действия что были проделаны совпадают с действиями проделанными в Теореме 1.1.
Но так как мы не знаем удобного способа преобразования \(\bold{n}\) 
в число мы проделали ту же самую работу но в скалярном виде.
Следовательно для любого биективного отношения
\(\cdot^* : \mathbb{V} \rightarrow ([0;|\Jt|] \subset \mathbb{Z} )\)
можно записать эту систему уравнений в матричном виде,
так как любое линейное преобразование можно записать в
матричном виде, обозначим:
\begin{equation}\begin{aligned}
\Q{\bold{v}\boldsymbol{C}_{0,0}}_{\bold{n}^*} =&
    r_1 \Q{\allExecutionRelativeIntensity{n}{k}} \seqStandartStar{\varv}{n} + \\
    &+\lambda\sum_{k=0}^{M-1}q_k \seqMinusStar{\varv}{n}{k} + \\
    &+\En r_0 \sum_{f=0}^{M-1} \mu_f (n_f + 1) \seqPlusStar{\varv}{n}{f} + \\
    &+r_1 \feedbackSum{f}{k} q_k \mu_f (n_f+1) \seqMinusPlusStar{\varv}{n}{k}{f}, \\
\Q{\bold{v}\boldsymbol{C}_{1,0}}_{\bold{n}^*} =
    &\overline{\En} \lambda \seqStandartStar{\varv}{n}+
    \En r_2 \sum_{f=0}^{M-1} \mu_f (n_f + 1) \seqPlusStar{\varv}{n}{f}, \\
\Q{\bold{v}\boldsymbol{K}}_{\bold{n}^*} =
    &\sum_{k=0}^{M-1}q_k\seqMinusStar{\varv}{n}{k}(u,t).
\end{aligned}\end{equation}
Очевидно что вид матрицы \(\boldsymbol{D}_0\) и \(\boldsymbol{I}\) 
будет соответствовать виду изложенному в предыдущей главе, тогда
Предыдущая система уравнений в матричном виде будет выглядить так:
\begin{equation}
    \pdv{\boldsymbol{H}(u,t)}{t} =
    \boldsymbol{H}(u,t)\bigg[\boldsymbol{C}_{0,0} + e^{ju}\boldsymbol{C}_{1,0} - \boldsymbol{D}_{0}\bigg]
    -j\sigma\pdv{\boldsymbol{H}(u,t)}{u}\bigg[e^{-ju}\boldsymbol{K} - \boldsymbol{I}\bigg]
\end{equation}
Что соответствует при \(\eta = \sigma\) Теореме 1.2, а следовательно получается такая система уравнений:
\begin{equation}\begin{aligned} \label{HFirstEquationParticular}
\pdv{\Hn(u,t)}{t}=
    &-\Q{\lambda+\allExecutionFullIntensity{n}{k}}\seqStandart{H}{n}(u,t)+\\
    &+j \sigma \En \pdv{\seqStandart{H}{n}(u,t)}{u}(i,t) + \\
    &+\overline{\En} \lambda e^{ju} \seqStandart{H}{n}(u,t)+\\
    &+r_1 \Q{\allExecutionRelativeIntensity{n}{k}} \seqStandart{H}{n}(u,t) + \\
    &+\lambda\sum_{k=0}^{M-1}q_k \seqMinus{H}{n}{k}(u,t) + \\
    &-j \sigma  e^{-ju} \sum_{k=0}^{M-1}q_k \pdv{\seqMinus{H}{n}{k}(u,t)}{u} + \\
    &+\En r_0 \sum_{f=0}^{M-1} \mu_f (n_f + 1) \seqPlus{H}{n}{f}(u,t) + \\
    &+\En r_2 e^{ju} \sum_{f=0}^{M-1} \mu_f (n_f + 1) \seqPlus{H}{n}{f}(u,t) + \\
    &+r_1 \feedbackSum{f}{k}
        q_k \mu_f (n_f+1) \seqMinusPlus{H}{n}{k}{f}(u,t),
\end{aligned}\end{equation}

\begin{equation}\begin{aligned}\label{HSecondEquationParticular}
\nJ \pdv{\Hn(u,t)}{t}=(e^{ju} - 1)
\Bigg\{
    &\lambda \nI \seqStandart{H}{n}(u,t)+\\
    &+j\sigma e^{-ju} \nG\pdv{\seqStandart{H}{n}(u,t)}{u} + \\
    &+r_2\sum_{f=0}^{M-1} \mu_f \nJ n_f \seqStandart{H}{n}(u,t)
\Bigg\}
.
\end{aligned}\end{equation}


