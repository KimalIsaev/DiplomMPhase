Используем замены \eqref{first_changes} в уравнении \eqref{vtoroe__uravnenie}:
\[\varepsilon \pdv{\boldsymbol{F}(w,\tau,\varepsilon)}{\tau} \boldsymbol{e} = 
(e^{jw\varepsilon} - 1)  \bigg[\boldsymbol{F}(w,\tau,\varepsilon) \boldsymbol{B}+
je^{-jw\varepsilon}  \pdv{\boldsymbol{F}(w,\tau,\varepsilon)}{w}\boldsymbol{I}   \bigg]\boldsymbol{e},\]
рассмотрим уравнение в пределе $\varepsilon \rightarrow 0$.
\[ \pdv{\boldsymbol{F}(w,\tau)}{\tau}\boldsymbol{e} = 
\bigg[\boldsymbol{F}(w,\tau,\varepsilon)\boldsymbol{B} +
j \pdv{\boldsymbol{F}(w,\tau,\varepsilon)}{w} \boldsymbol{I}  \bigg]\boldsymbol{e}.\]
После заменим решение $F(w,\tau)=Re^{jwx(\tau)}$:
\begin{align}\label{a(x)}
	x'(\tau)=\boldsymbol{rBe}-x(\tau)\boldsymbol{rIe}=a(x).
\end{align}

\begin{comment}
Откуда получам, что:
\[x(\tau) = ce^{-\boldsymbol{\overline{E}KR}\tau} + \frac{\boldsymbol{\overline{E}C_{1,0}R}}{\boldsymbol{\overline{E}KR}}\]
%\[M{}(Re^{jwx(\tau)})_{w = 0}' = jx(\tau)R \]
\[c \in [-\frac{\boldsymbol{\overline{E}C_{1,0}R}}{\boldsymbol{\overline{E}KR}};+\infty) \subset \mathbb{R},
\text{ потому что }\]\[\frac{x(\tau)}{\sigma}\text{- среднее количество заявок на орбите, а оно должно быть неотрицатеоьным.}\]
Если 
\[c \in [-\frac{\boldsymbol{\overline{E}C_{1,0}R}}{\boldsymbol{\overline{E}KR}};0] \subset \mathbb{R}, \text{ то }
x \in [\frac{\boldsymbol{\overline{E}C_{1,0}R}}{\boldsymbol{\overline{E}KR}}-|c|;
\frac{\boldsymbol{\overline{E}C_{1,0}R}}{\boldsymbol{\overline{E}KR}}]\subset \mathbb{R}\]
Если
\[c \in (0;+\infty) \subset \mathbb{R}, \text{ то }
x \in (\frac{\boldsymbol{\overline{E}C_{1,0}R}}{\boldsymbol{\overline{E}KR}};
\frac{\boldsymbol{\overline{E}C_{1,0}R}}{\boldsymbol{\overline{E}KR}}+c]\subset \mathbb{R}\]
Удостоверся, что правильно решил дифференциальное уравнение.
\end{comment}
