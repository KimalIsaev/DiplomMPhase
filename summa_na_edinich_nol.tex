Определим \(\boldsymbol{e}\) как вектор-столбец из единиц размерности \(|\mathbb{J}|\), 
а \(0\) как вектор-строка из нулей размерности \(|\mathbb{J}|\) и покажем утверждение:
\begin{equation} \label{summa_na_edinich_nol}
	\bigg[\sum_{q = -1}^{W} \boldsymbol{C}_{q,k}-\boldsymbol{D}_k\bigg]\boldsymbol{e}=0.
\end{equation}
Оно пригоится нам в дальнейшем. \(\forall n \in [1, |\mathbb{J}|] \subset \mathbb{Z}\):
\[\sum_{q = -1}^{W}\sum_{\nu = 1, \atop \tilde{\nu}\neq y^*\& q\neq 0}^{|\mathbb{J}|} C^{(q)}_{\tilde{n}, \tilde{\nu},k}-
\sum_{\nu = 1}^{|\mathbb{J}|}\sum_{q = -1 ,\atop \tilde{\nu}\neq y^*\& q\neq 0 }^{W}C^{(q)}_{\tilde{n}, \tilde{\nu},k} = 0.\]
Следовательно, каждый элемент итогового стобца равен нулю, а значет сам вектор-столбец нулевой.
\begin{comment}
Благодаря утверждению \eqref{summa_na_edinich_nol} можем преобразовать  \eqref{uravnenie_isaeva}:
\[\pdv{\boldsymbol{H}(u,t)}{t}\overline{\boldsymbol{E}}=\boldsymbol{H}(u,t)(e^{ju}-1)\]
\begin{align*}
\pdv{\boldsymbol{H}(u,t)}{t}\boldsymbol{e}=&\boldsymbol{H}(u,t)(e^{ju}-1)\boldsymbol{B}\boldsymbol{e}+j\sigma e^{-ju}\frac{\partial \boldsymbol{H}(u,t)}{\partial u}(e^{ju}-1)\boldsymbol{I_{0}}\boldsymbol{e}=\\
&=(e^{ju}-1)\{\boldsymbol{H}(u,t)\boldsymbol{B}\boldsymbol{e}+j\sigma e^{-ju}\frac{\partial \boldsymbol{H}(u,t)}{\partial u}\boldsymbol{I_{0}} \boldsymbol{e}\}
\end{align*} 
\end{comment}
