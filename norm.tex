Введём множество \(\mathbb{V}\):
\[ \mathbb{V} = \{(x_0, x_1 ... x_{n-1}) | \forall k \in [0;n) \subset \mathbb{Z}, x_k \in [0; M) \subset \mathbb{Z}\}.\]
Биективное отношение
\begin{equation}\label{number_function}
\cdot^* : \mathbb{V} \rightarrow ([0;M^n-1] \subset \mathbb{Z})
\end{equation}
будем называть функцией нумеровки.\\
Функцию нумеровки определённую по формуле:
\begin{equation}\label{number_function}
\bold{x}^* = \sum_{k=0}^{n-1}x_k M^k,
\end{equation}
будем называть циферной нумеровочной функцией.
Так как представляем, что каждый элемент вектора является цифрой
в числе с основанием M.
