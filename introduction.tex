\hspace*{\parindent}%
Системы массового обслуживания с орбитами, также называемые RQ-системы, обладают большой популярностью и широко рассмотрены в литературе [6, 8, 12, 13]. 

Очень похожая работа была также рассмотрена в [20]. В ней, так же как и в этой работе, гиперэкспаненциальное время обслуживания и, если заявка пришла в тот момент, когда все приборы заняты, то она так же отправляется на орбиту, где ожидает время, распределённое по экспоненциальному закону. Однако, в данной работе рассматривается гиперэкспоненциальное распление с M фазамаи, а не с двумя, и после завершения обслуживания, в работе [20] заявка покидает в систему, в то время как в данной работе заявка может также уйти на орбиту или же мгновенно перейти на повторное обслуживание. И так же, с помощью асимптотически диффузионного анализа был найден ряд распределения количества заявок на орбите.

В данной работе рассматриваются M-фазные системы $M|H_M|N$ с обратной связью.

В первой главе


Но в нашем исследовании намного больше пригодилась статья [21].

Также помогли книги [1,2,3,5,9,10,11] для ознакомления с различными методами.

\textbf{Цель дипломной работы:} построить ряд распределения, или его апроксимацию, для количества заявок на орите для RQ-системы $M|H_M|N$ в стационарном режиме.

\textbf{Задачи:}\\
1. Построить математическую модель систем $M|H_M|N$ с обратной связью.\\
2. Составить систему дифференциальных уравнений Колмогорова для систем $M|H_M|N$ с обратной связью.\\
3. С помощью метода асимптотического анализа найти коэффициенты переноса и диффузии дифференциальных уравнений систем $M|H_M|N$ с обратной связью.\\
4. С помощью метода асимптотически диффузионного анализа вычислить плотность распределений вероятностей произвольного количества заявок на орбите и получить дискретные распределения вероятностей.