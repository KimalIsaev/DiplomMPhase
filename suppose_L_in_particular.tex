Предположим вид \(\Ln\), после чего докажем верность 
данного предположения подставкой в уравнения и получением тождества, 
где \(\Rn\) будет определяться как:
\[\Rn = \frac{\Ln}{\sum_{\bar{m}}^{\Jt}\Lm},\]
для соблюдения \eqref{RSumEqualsToOne}.

%может стоит двоиточее внитри круглых заменить на пробел\)
Пусть в \(\bar{n}, \forall d \in \Mt: n_d \in \Nt \), тогда:
\begin{equation} \label{firstLDefinition}
    \Ln = \Ldefinition{n}{d}.
\end{equation}
Если \(\exists! k \in \Mt \),что \(n_k=-1\), в то время как 
    \(\forall d \neq k \in \Mt: n_d \in \Nt\), тогда:
\begin{equation} \label{secondLDefinition}
    \Ln = \LminusDefinition{m}{d}{k},
\end{equation}
где \(\forall d \neq k \in \Mt: m_d = n_d\), но \(m_k = n_k + 1 = 0\),
следовательно в таком случае \(\Ln = 0\).

Если же у нас есть лишь подозрения, что 
\(\exists! k \in \Mt \),что \(n_k=-1\), в то время как 
\(\forall d \neq k \in \Mt: n_d \in \Nt\), 
и нам известно какое именно k подозревать.
То мы все равно можем использовать \eqref{secondLDefinition} так как, 
если \(\forall d \in \Mt: n_d \in \Nt\) и \(k \in \Mt\):

\begin{equation}\begin{aligned}
    \Ln &= \LminusDefinition{m}{d}{k} = \\
        &= \Ldefinition{n}{d},
\end{aligned}\end{equation}
где \(\forall d \neq k \in \Mt: m_d = n_d\), но \(m_k = n_k + 1\).
 
