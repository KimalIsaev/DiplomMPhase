\begin{comment}
Перейдём к частному случаю, системы с n фазами M приборами и
источником повторных вызовов (M|\(M_n\)|M|ИПВ)
\[W=1, N = 1\]

Множество \(\mathbb{J}\) определим как:
\[\mathbb{J}=\{\bold{y}:\sum_{k=0}^{n-1}y_k	\leq M\}\]
А коэффиценты полиномов прехода определим так:
\end{comment}
\begin{urv}{60}
	1.&\text{Если }\exists \bold{z}, \bold{y} \in \mathbb{V}, \exists l,i \in [0, +\infty) \subset \mathbb{Z}, 
	\exists k \in [0,n-1] \subset \mathbb{Z}, \forall m \neq k  \in [0,n-1] \subset \mathbb{Z} & \\ 
	 &\text{ такие, что } (z_k-y_k=1)\&(z_m-y_m=0)\&(l-i=0), & \\ 
	 &\text{ то } C_{\bold{y}^*,\bold{z}^*, 0} = q_k \lambda, C_{\bold{y}^*,\bold{z}^*, 1} = 0  &
	\\ \\
	2.&\text{Если }\exists \bold{z}, \bold{y} \in \mathbb{V}, \exists l,i \in [0, +\infty) \subset  \mathbb{Z}, 
	\exists k \in [0,n-1] \subset \mathbb{Z}, \forall m \neq k  \in [0,n-1] \subset \mathbb{Z} & \\  
	&\text{ такие, что } (z_k-y_k=1)\&(z_m-y_m=0)\&(l-i=-1), & \\ 
	&\text{ то } C^{(-1)}_{\bold{y}^*,\bold{z}^*, 0} = 0, C^{(-1)}_{\bold{y}^*,\bold{z}^*, 1} =  q_k \sigma &
	\\ \\
	3.&\text{Если }\exists \bold{z}, \bold{y} \in \mathbb{V}, \exists l,i \in [0, +\infty) \subset  \mathbb{Z}, 
	\forall m  \in [0,n-1] \subset \mathbb{Z} & \\  
	&\text{ такие, что } (z_m-y_m=0)\&\bigg(\sum_{k=0}^{n-1} z_k= M - 1\bigg)\&(l-i=1), & \\ 
	&\text{ то } C^{(1)}_{\bold{y}^*,\bold{z}^*, 0} = \lambda, C^{(1)}_{\bold{y}^*,\bold{z}^*, 1} = 0  &
	\\ \\
	4.&\text{Если }\exists \bold{z}, \bold{y} \in \mathbb{V}, \exists l,i \in [0, +\infty) \subset  \mathbb{Z}, 
	\exists k \in [0,n-1] \subset \mathbb{Z}, \forall m \neq k  \in [0,n-1] \subset \mathbb{Z} & \\ 
	&\text{ такие, что } (z_k-y_k=-1)\&(z_m-y_m=0)\&(l-i=0), & \\ 
	&\text{ то } C^{(0)}_{\bold{y}^*,\bold{z}^*, 0} = \mu_k r_0, C^{(0)}_{\bold{y}^*,\bold{z}^*, 1} = 0  &
	\\ \\
	5.&\text{Если }\exists \bold{z}, \bold{y} \in \mathbb{V}, \exists l,i \in [0, +\infty) \subset  \mathbb{Z}, 
	\exists k \in [0,n-1] \subset \mathbb{Z}, \forall m \neq k  \in [0,n-1] \subset \mathbb{Z} & \\ 
	&\text{ такие, что } (z_k-y_k=-1)\&(z_m-y_m=0)\&(l-i=1), & \\ 
	&\text{ то } C^{(1)}_{\bold{y}^*,\bold{z}^*, 0} = \mu_k r_2, C^{(1)}_{\bold{y}^*,\bold{z}^*, 1} = 0  &
	\\ \\
	6.&\text{Если }\exists \bold{z}, \bold{y} \in \mathbb{V}, \exists l,i \in [0, +\infty) \subset  \mathbb{Z}, 
	\exists k, p \in [0,n-1] \subset \mathbb{Z}, k \neq p, \forall m \in [0,n-1] \subset \mathbb{Z},  & \\ 
	&(m \neq k) \& (m \neq p) \text{ такие, что } (z_k-y_k=-1)\&(z_p-y_p=1)\&(z_m-y_m=0)\&(l-i=1), & \\ 
	&\text{ то } C^{(0)}_{\bold{y}^*,\bold{z}^*, 0} = \mu_k r_1 q_p, C^{(0)}_{\bold{y}^*,\bold{z}^*, 1} = 0  &
\end{urv}
