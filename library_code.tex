Подключаем все необходимые библиотеки:
sympy для символьного исчисления,
numpy для работы c матрицами: 
\begin{minted}{Python}
from sympy import Matrix
import numpy as np
\end{minted}

Определим функцию для вычисления списка из функции 
для одних и тех же аргументов:
\begin{minted}{Python}
def value_eval(q, y, z, arg_dict, temp_arg_dict, value_function_lst):
	return [f(q, y, z, arg_dict, temp_arg_dict) for f in value_function_lst]
\end{minted}

Определим функцию для вычисления списка из функции
для одних и тех же аргументов,
только если проверочная функция не равна нулю,
иначе отдаём пустой список, так как значение проверочной функции
говорит о том что вычислять ничего не надо.
\begin{minted}{Python}
def eval_value_function_list_if_check(
		q, y, z, const_arg, arg, check_f, value_f_lst):
	if check_f(q, y, z, const_arg, arg):
		return value_eval(q, y, z, const_arg, arg, value_f_lst)
	else:
		return []
\end{minted}

Определим функцию, которая вычисляет список из функций, где
первая элесент является проверочной функцие, а все оставшиеся
значимыми.
Если же список пуст, то выдаём обратно пустой список:
\begin{minted}{Python}
def eval_function_list(q, y, z, const_arg, arg, function_lst):
	if (len(function_lst) > 1):
		check_f = function_lst[0]
		value_f_lst = function_lst[1:]
		return eval_value_function_list_if_check(q, y, z,
			const_arg, arg, check_f, value_f_lst)
	else:
		return []
\end{minted}

Определим функцию, что вычисляет список из списков функций(LFL)
с локальными аргументами, где локальные аргументы, это
те что могут поменятся с каждым  измением 
состояния системы массового обслуживания.
Причём в этом списке только лишь один элемент
иммет проверучную функцию, что не равна нулю, так как
система массового обслуживания может находится
одновременно только в одном единственном состоянии,
а проверочная функция проверяет в каком именно 
состоянии она находится:
\begin{minted}{Python}
def eval_LFL_with_local_arg(q, y, z, const_arg, LFL, local_arg):
	for f in LFL:
		e = eval_function_list(q, y, z, const_arg, local_arg, f)
		if e:
			return e
	return []
\end{minted}

Определим функцию, что вычисляет список из списков функций(LFL),
вычислив перед этим из всех остальных аргументов, что у нас имеются,
локальные. Если же не имеется изменение состояния, или хотя бы
изменения орбиты, в таком случае возвращаем пустой список.
\begin{minted}{Python}
def eval_LFL(q, y, z, const_arg, LFL, create_arg):
	if (q == 0) and (y == z):
		return []
	else:
		local_arg = create_arg(q, y, z, const_arg)
		return eval_LFL_with_local_arg(q, y, z,
			const_arg, LFL, local_arg)
\end{minted}

Заполняем вектор меньшей длины нулями
до необходимой длины:
\begin{minted}{Python}
def fill_to_standart_size(lst, n):
	return lst + [0]*(n-len(lst))
\end{minted}

Создаём матрицу из векторов, что состоят из коэфицентов для полиномов,
которые зависят от количества заявок на орбите:
\begin{minted}{Python}
def create_state_matrix_with_polynom_inside(
		q, J, N, const_arg, list_of_function_list, create_arg):
	return [[fill_to_standart_size(
		eval_LFL(q, y, z, const_arg, list_of_function_list, create_arg),
				N) for z in J] for y in J]
\end{minted}

Переворачиваем матрицу так, что бы полином внутри вышел наружу,
таким образом у нас полуается вектор из матриц, что являются
коэфицентами для полинома, который зависит от количества заявок на орбите:
\begin{minted}{Python}
def transpose_state_matrix_inside(Mq):
	return [Matrix(m) for m in np.transpose(Mq, axes=(2, 0, 1)).tolist()]
\end{minted}

Создаём вектор из вектор из матриц, что являются
коэфицентами для полинома, который зависит от количества заявок на орбите.
\begin{minted}{Python}
def create_orbit_change_row(
		q, J, N, const_arg, list_of_function_list, create_arg):
	return transpose_state_matrix_inside(
		create_state_matrix_with_polynom_inside(
			q, J, N, const_arg, list_of_function_list, create_arg))
\end{minted}

Создаём конечную матрицу из матриц \(\boldsymbol{C}_{q.k}\). 
\begin{minted}{Python}
def algorithm(J, N, W, const_arg, list_of_function_list,
		create_arg=lambda q, z, y, arg_dict: {}):
	return [create_orbit_change_row(
			q, J, N+1, const_arg, list_of_function_list, create_arg)
		for q in range(-1, W + 1)]
\end{minted}

