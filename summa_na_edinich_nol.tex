Домножив матричное уравнение на единичный вектор-столбец $\boldsymbol{e}$ и,
с учетом $$(\boldsymbol{C} + \boldsymbol{S} - \boldsymbol{D}) \boldsymbol{e} = 0$$ 
и $$(\boldsymbol{K} - \boldsymbol{I}) \boldsymbol{e} = 0$$ получим
\begin{equation} \label{vtoroe__uravnenie}
\begin{aligned} 
    \frac{\partial \boldsymbol{H}(u,t)}{\partial t}\boldsymbol{e}=
        &\boldsymbol{H}(u,t)(e^{ju}-1)\boldsymbol{S}\boldsymbol{e}
            +j\eta e^{-ju}\frac{\partial \boldsymbol{H}(u,t)}{\partial u}(e^{ju}-1)\boldsymbol{I}\boldsymbol{e}=\\
        &=(e^{ju}-1)\Bigg[\boldsymbol{H}(u,t)\boldsymbol{S}
            +j\eta e^{-ju}\frac{\partial \boldsymbol{H}(u,t)}{\partial u}\boldsymbol{I} \Bigg]\boldsymbol{e}.
\end{aligned}
\end{equation}
\begin{comment}
Благодаря утверждению \eqref{summa_na_edinich_nol} можем преобразовать  \eqref{uravnenie_isaeva}:
\[\pdv{\boldsymbol{H}(u,t)}{t}\overline{\boldsymbol{E}}=\boldsymbol{H}(u,t)(e^{ju}-1)\]
\begin{align*}
\pdv{\boldsymbol{H}(u,t)}{t}\boldsymbol{e}=&\boldsymbol{H}(u,t)(e^{ju}-1)\boldsymbol{B}\boldsymbol{e}+j\sigma e^{-ju}\frac{\partial \boldsymbol{H}(u,t)}{\partial u}(e^{ju}-1)\boldsymbol{I_{0}}\boldsymbol{e}=\\
&=(e^{ju}-1)\{\boldsymbol{H}(u,t)\boldsymbol{B}\boldsymbol{e}+j\sigma e^{-ju}\frac{\partial \boldsymbol{H}(u,t)}{\partial u}\boldsymbol{I_{0}} \boldsymbol{e}\}
\end{align*} 
\end{comment}
