\thispagestyle{empty}
\begin{center}
	Министерство науки и высшего образования Российской Федерации\\
	НАЦИОНАЛЬНЫЙ ИССЛЕДОВАТЕЛЬСКИЙ\\
	ТОМСКИЙ ГОСУДАРСТВЕННЫЙ УНИВЕРСИТЕТ (НИ ТГУ)\\
	Институт прикладной математики и компьютерных наук\\
	Кафедра теории вероятностей и математической статистики\\[30pt]
\end{center}

\noindent\rule{105mm}{0pt}УТВЕРЖДАЮ\\
\rule{105mm}{0pt}Руководитель ООП\\
\rule{105mm}{0pt}д-р техн. наук, профессор\\
\rule{105mm}{0pt}$\underset{\text{подпись}}{\underline{\hspace{0.2\textwidth}}}$ А.М. Горцев\\
\rule{105mm}{0pt}<<\rule{10mm}{0.4pt}>>\rule{25mm}{0.4pt} 2020 г.\\

\begin{center}
	ЗАДАНИЕ
\end{center}
\hspace*{\parindent}%
	по выполнению выпускной квалификационной работы бакалавра
	студентки Исаеву Кимал Илхам оглы группы \textnumero 931820, «Прикладная Математика и Информатика», «Институт Прикладной Математики и Компьютерных Наук».

1. Тема выпускной квалификационной работы\\
<<Исследование многофазной RQ-системы массового обслуживания с общей орбитой.>>\\


2. Срок сдачи обучающимся выполненной выпускной квалификационной работы:

а) в учебный офис / деканат -- 09.06    \;  б) в ГЭК -- 09.06

3. Исходные данные к работе:\\
\hspace*{\parindent}%
Объектом исследования выступают RQ-система вида $M|H_M|N$ с обратной связью.

Предметом исследования выступают ряд распределения вероятностей RQ-систем вида $M|H_M|N$ с обратной связью.

Целью настоящей работы является асимптотически-диффузионный анализ RQ-систем вида $M|H_M|N$ с обратной связью.

Задачи:

- изучение литературы по исследованию RQ-систем с обратной связью;

- построение математических модели RQ-системы вида $M|H_M|N$ с обратной связью;

- построене системы дифференциальных уравнений Колмогорова для системы $M|H_M|N$ с обратной связью;

- построение аппроксимаций для распределения вероятностей числа приборов, находящихся на первой и второй фазе в системе вида $M|H_M|N$ с обратной связью методом асимптотического анализа и в предельном условиии большой задержки заявок на орбите;

- построение аппроксимаций распределения вероятностей количества заявок на орбите в RQ-системе вида $M|H_M|N$ с обратной связью методом асимптотически-диффузионного анализа;

Методами исследования являются методы теории вероятностей, теории случайных процессов, теории массового обслуживания, теории дифференциальных уравнений, а также методы асимптотического и асимптотически диффузионного анализа.

\thispagestyle{empty}
4. Краткое содержание работы\\
\hspace*{\parindent}%
Основные разделы:
\begin{itemize}
	\item[I.] Исcледование марковской системы массового обслуживания с постоянными интенсивностями перехода между состояниями, за исключением интенсивности ухода заявок с орбиты, с общей орбитой, на которую заявки приходят и с которой заявки уходят ординарно, в условиях стремящейся к бесконечности задержки на орбите.
	\item[II.] Исследование M|$H_M$|N с общей орбитой.
	\item[III.] Нахождение программными средствами аппроксимации распределения заяв
ок на орбите.
\end{itemize}


\noindent Научный руководитель ВКР\\
\noindent д-р техн. наук, профессор,\\
\noindent заведующая каф. ТВиМС\rule{30mm}{0pt}\rule{30mm}{0.4pt}\rule{30mm}{0pt} С. П. Моисеева.\\

\noindent Задание принял к исполнению\rule{10mm}{0pt}<<\rule{10mm}{0.4pt}>>\rule{25mm}{0.4pt} 2020 г.\rule{10mm}{0pt}\rule{23mm}{0.4pt}\\
